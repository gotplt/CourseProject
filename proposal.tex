\documentclass{report}
\usepackage{fullpage}
\begin{document}
\title{ManBearPig. An attempt at providing native full text search to the mandoc system.}
\author{spiros thanasoulas st19@illinois.edu}
\maketitle

\section*{Description}
The goal of this project would be to create a report and possibly code improvements towards providing
a backend that supports Full Text Search capabilities for the mandoc project (https://mandoc.bsd.lv/).

\section*{Background}
UNIX system provide their documentation to the user through a set of tools collectively referred to
as the Manual Page system. The well known man(1) command exists today on all UNIX systems but even
on other platforms like MacOSX and android. Searching efficiently keywords and semantics has been
of paramount importance for the user to quickly get to the relevant manual page and the command
apropos(1) traditionally served that purpose, meaning doing database lookups. 
The databases are built with the makewhatis(1) tool.

\section*{Project proposal}
We will investigate the C source code of the mandoc project, targeted on the modules of searching
and database creation. The goal of this project would be to lay a path for full text search capabilities
from the apropos command. Currently only certain words of a manual page are indexed and their semantic
information stored with them, in a persisted to a file database that on the outerlevel is implemented as 
a hash map.

To allow for the full text search capabilities we will implement a database based on trigrams keying an
inverted index  of the full text being contained in a manual page, 
after it has been parsed from the mdoc parsers and only the content remains.

In detail the goal of the project would be to create the equivalent database of the makewhatis(1) db that
is currently created, but which stores the trigrams. Due to lack of time no optimizations for very large
databases are going to be implemented and the testing input will be constrained enough to make sure 
the datastructures will be able to fit in memory.

The generated database will be evaluated by dumping the contents and making sure all the trigrams that should
be produced and only those are contained within it. A test harness to ensure that will be provided.

If time permits the search capabilities will be attempted to connect to the database through the apropos
command and query for a text string.

{\it note to the reviewer: although i would love to finish the whole thing but it might be unfeasable in 
around 25hrs that i have budgeted it for it. My intention though is to lay the foundation so that a patch
will be eventually merged in the mandoc codebase, not to demo something that noone will ever user}

\section*{Proposed Workflow}
We propose that the analysis and development will be split across 5 6hr man - days of work

\begin{center}
\begin{tabular}{| c | c |}
\hline
Day 1 & Code and Documentation analysis.\\
& Reviewing the makewhatis utility and the resulting databases it creates \\
& examine the relevant code flow and find where to plugin the new functionality.\\
\hline
Day 2 & Design of the binary file format that will store\\
& to the trigrams data structure, as well as the parsing\\
& functions to extract them. \\
\hline
Day 3 & Development and Documentation  \\
\hline
Day 4 & Development, Documentation and test harness \\
\hline
Day 5 & Final report \\
\hline
\end{tabular}
\end{center}

\section*{members}
st19 / solo project 
\end{document}
